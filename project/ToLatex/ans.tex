\documentclass{article} 
\usepackage{amsmath}
\usepackage[UTF8]{ctex}
	\title{}\date{} \linespread{1}
\usepackage{ntheorem}
\usepackage{amssymb}
\usepackage{graphicx}
\usepackage{geometry} 
	\geometry{a4paper,left=2cm,right=2cm,top=2cm,bottom=2.5cm}
\usepackage{float}
\usepackage{paralist}
	\let\itemize\compactitem
	\let\enumerate\compactenum
	\setlength{\parindent}{0pt}

\newcommand{\env}[2]{\begin{#1}#2\end{#1}}
\renewcommand{\P}{\mathbb P}
\renewcommand{\E}{\mathbb E}


\usepackage{listings}
\usepackage{ctex}
\lstset{
    basicstyle          =   \sffamily,
    keywordstyle        =   \bfseries,
    commentstyle        =   \rmfamily\itshape,
    stringstyle         =   \ttfamily,
    flexiblecolumns,
    numbers             =   left,
    showspaces          =   false,
    numberstyle         =   \zihao{-5}\ttfamily,
    showstringspaces    =   false,
    captionpos          =   t,
    frame               =   lrtb,
    breaklines      =   true,
    columns         =   fixed,
    basicstyle      =   \zihao{-5}\ttfamily,
    numberstyle     =   \zihao{-5}\ttfamily,
    stringstyle     =   \color{magenta},
    commentstyle    =   \color{red}\ttfamily,
}

\begin{document}


\par
\begin{enumerate}\par
\item 多项式函数\par
	\textbf{Define. }\par
		\begin{align*}
f\left(x\right) &= \sum\limits_{i=0}^{n} a_i x^i  \tag{一元}\\
f\left(\boldsymbol x\right) &= \sum\limits_{\boldsymbol i=\left(0,...,0\right)_n}^{\left(\dim,...,\dim\right)_n} \left(a_{\boldsymbol i} · \prod\limits_{i_j \in \boldsymbol i, x_0 = 1}x_{i_j}\right)
\end{align*}

	\textbf{Include. }\par
		\begin{enumerate}\par
		\item 线性函数 (一次函数)\par
			\textbf{Define. }\par
				\begin{align*}
f\left(x\right) &= a x + b \tag{一元一次}\\
f\left(\boldsymbol x\right) &= \boldsymbol A \boldsymbol x + \boldsymbol b  \tag{多元一次}
\end{align*}

			\textbf{Property. }\par
				\begin{enumerate}\par
				\item 线性函数的零点集\par
					\begin{itemize}
					\item 解线性方程\par
						\textbf{Problem. }\par
							\begin{align*}\boldsymbol A \boldsymbol x = \boldsymbol b\end{align*}

							知$\boldsymbol A, \boldsymbol b$ 求 $\boldsymbol x$.\par
						\textbf{Property. }\par
							\begin{itemize}
							\item 解的存在性\par
								\begin{align*}
\left\{\begin{matrix}
rank\left(\left(\boldsymbol A\ \boldsymbol b\right)\right) = rank\left(\boldsymbol A\right) \quad \text{有解}\\
rank\left(\left(\boldsymbol A\ \boldsymbol b\right)\right) > rank\left(\boldsymbol A\right) \quad \text{无解}\\
\end{matrix}\right.
\end{align*}

								\begin{itemize}
								\item 有解\par
									\begin{itemize}
									\item 有唯一解\par
										\begin{align*}\boldsymbol x = \boldsymbol A^{-1} \boldsymbol b\end{align*}

										.$\boldsymbol A$是方阵且非奇异\par
										\begin{align*}\boldsymbol A \in C^{n\times n}, |\boldsymbol A| \neq  0\end{align*}

									\item 有无穷个解\par
										通解 $\boldsymbol x = \boldsymbol A^{-^{\left\{1\right\}}} \boldsymbol b + \left(\boldsymbol I - \boldsymbol A^{-^{\left\{1\right\}}} A\right) c$\par
										特解 $\boldsymbol x = \boldsymbol A^{-^{\left\{1\right\}}} \boldsymbol b$\par
										.$\boldsymbol A^{-1}$不存在或无意义\par
										\begin{itemize}
										\item 极小范数解\par
											\begin{align*}
\min_{\boldsymbol x} \quad& \| \boldsymbol x\| _2\\
s.t. \quad& \boldsymbol A \boldsymbol x = \boldsymbol b
\end{align*}

											\begin{align*}\boldsymbol x = \boldsymbol A^{-^{\left\{1,3\right\}}} \boldsymbol b\end{align*}

											极小范数解唯一。\par
										\end{itemize}
									\end{itemize}
								\item 无解\par
									最小二乘解\par
										\begin{align*}\min_{\boldsymbol x} \quad& \| \boldsymbol A \boldsymbol x - \boldsymbol b\| _2\end{align*}

										\begin{align*}\tilde{\boldsymbol x} = \boldsymbol A^{-^{\left\{1,4\right\}}} \boldsymbol b\end{align*}

									极小最小二乘解\par
										\begin{align*}
\min_{\boldsymbol x} \quad& \| \boldsymbol A \boldsymbol x - \boldsymbol b\| _2\\
\min_{\boldsymbol x} \quad& \| \boldsymbol x\| _2
\end{align*}

										\begin{align*}\tilde{\boldsymbol x} = \boldsymbol A^+ \boldsymbol b\end{align*}

								\end{itemize}
							\end{itemize}
					\end{itemize}
				\end{enumerate}\par
		\item 二次函数\par
			\textbf{Define. }\par
				\begin{align*}
f\left(x\right) &= a x^2 + b x + c  \tag{一元二次}\\
f\left(\boldsymbol x\right) &= \boldsymbol x^T \boldsymbol A \boldsymbol x + \boldsymbol b \boldsymbol x + c  \tag{多元二次, 矢量式}\\
 &= \sum\limits_{i=1}^{\dim} \sum\limits_{j=1}^{\dim} a_{ij} · x_i x_j + \sum\limits_{i=1}^{\dim} b_i x_i + c  \tag{分量式}
\end{align*}

			\textbf{Property. }\par
				\begin{enumerate}\par
				\item 二次函数的零点集 (二次曲面)\par
					\textbf{Define. }\par
						二次曲面:\par
						\begin{align*}
&\left\{ \boldsymbol x \ |\ \boldsymbol x^T \boldsymbol A \boldsymbol x + \boldsymbol b \boldsymbol x + c = 0\right\}\\
\Leftrightarrow &\left\{ \boldsymbol x' \ |\ \boldsymbol x'^T \boldsymbol A' \boldsymbol x' = 0 \ |\  \boldsymbol x' = \left(\begin{matrix} \boldsymbol x \\ 1 \end{matrix}\right)\right\}
\end{align*}

					\begin{itemize}
					\item 解二次方程\par
						\textbf{Problem. }\par
							(input) $a, b, c$\par
							求输入二次函数的零点集$\left\{x\right\}$\par
							\begin{align*}f\left(x\right) = a x^2 + b x + c = 0\end{align*}

						\textbf{Algorithm. }\par
							\begin{itemize}
							\item 一元二次\par
								\begin{align*}
x = \frac{- b \pm \sqrt{\Delta }}{2 a}\\
\Delta  = b^2 - 4 a c
\end{align*}

								\begin{itemize}
								\item 解的性质\par
									\begin{itemize}
									\item $\Delta  > 0$, 两个实数根\par
									\item $\Delta  = 0$, 一个实数二重根\par
									\item $\Delta  < 0$, 两个复数根\par
									\end{itemize}
								\end{itemize}
							\end{itemize}
					\end{itemize}
					\textbf{Include. }\par
						\begin{enumerate}\par
						\item Euclid球\par
							\textbf{Define. }\par
								令$\boldsymbol A = \boldsymbol I, \boldsymbol b = \boldsymbol 0, c = -r^2$\par
								\begin{align*}
&\left\{ \boldsymbol x \ |\ \| \boldsymbol x - \boldsymbol x_c\| _2 \le  r, r>0\right\}\\
\Leftrightarrow &\left\{ \boldsymbol x \ |\ \left(\boldsymbol x - \boldsymbol x_c\right)^T \left(\boldsymbol x - \boldsymbol x_c\right) \le  r^2, r>0\right\}  \tag{等价}\\
\Leftrightarrow &\left\{ \boldsymbol x_c + r \boldsymbol u \ |\ \| \boldsymbol u\| _2 \le  r, r>0\right\}  \tag{等价}
\end{align*}

								距离中心点$\boldsymbol x_c$为恒定值$r$的点的集合.\par
							\textbf{Property. }\par
								\begin{itemize}
								\item 是凸集\par
								\end{itemize}
						\item 椭球\par
							\textbf{Define. }\par
								令$\boldsymbol A = \boldsymbol P^{-1}, \boldsymbol b = \boldsymbol 0, c = -1$是正定矩阵.\par
								\begin{align*}
&\left\{ \boldsymbol x \ |\ \left(\boldsymbol x - \boldsymbol x_c\right)^T P^{-1} \left(\boldsymbol x - \boldsymbol x_c\right) \le  1, \boldsymbol P = \boldsymbol P^T \succeq  0\right\}\\
\Leftrightarrow &\left\{ \boldsymbol x_c + A \boldsymbol u \ |\ \| \boldsymbol u\| _2 \le  1\right\}  \tag{等价}
\end{align*}

							\textbf{Property. }\par
								\begin{itemize}
								\item 是凸集\par
								\end{itemize}
						\item 双曲体\par
							\textbf{Define. }\par
								令$\boldsymbol A$是非正定矩阵.\par
						\item 柱体\par
							\textbf{Define. }\par
						\end{enumerate}\par
				\end{enumerate}\par
		\item 三次函数\par
			\textbf{Define. }\par
			\textbf{Property. }\par
				\begin{enumerate}\par
				\item 三次函数的零点集 (三次曲面)\par
					\textbf{Define. }\par
						\begin{align*}
f\left(x\right) &= \sum\limits_{i=0}^{3} a_i x^i  \tag{一元}
\end{align*}

					\begin{itemize}
					\item 解三次方程\par
						\textbf{Problem. }\par
						\textbf{Algorithm. }\par
							\begin{itemize}
							\item 一元三次\par
							\end{itemize}
					\end{itemize}
				\end{enumerate}\par
		\item 四次函数\par
			\textbf{Define. }\par
				\begin{align*}
f\left(x\right) &= \sum\limits_{i=0}^{4} a_i x^i  \tag{一元}\\
f\left(\boldsymbol x\right) &= \sum\limits_{i_1=1}^{\dim} \sum\limits_{i_2=1}^{\dim} \sum\limits_{i_3=1}^{\dim} \sum\limits_{i_4=1}^{\dim} a_{i_1 i_2 i_3 i_4} · x_{i_1} x_{i_2} x_{i_3} x_{i_4} \\
&+ \sum\limits_{i_1=1}^{\dim} \sum\limits_{i_2=1}^{\dim} \sum\limits_{i_3=1}^{\dim} b_{i_1 i_2 i_3} · x_{i_1} x_{i_2} x_{i_3} \\
&+ \sum\limits_{i_1=1}^{\dim} \sum\limits_{i_2=1}^{\dim} c_{i_1 i_2} · x_{i_1} x_{i_2} \\
&+ \sum\limits_{i_1=1}^{\dim} d_{i_1} x_{i_1} \\
&+ e  \tag{多元, 分量式}
\end{align*}

			\textbf{Property. }\par
				\begin{enumerate}\par
				\item 四次函数的零点集 (四次曲面)\par
					\textbf{Define. }\par
					\begin{itemize}
					\item 解四次方程\par
						\textbf{Problem. }\par
						\textbf{Algorithm. }\par
							\begin{itemize}
							\item 一元四次\par
							\end{itemize}
					\end{itemize}
					\textbf{Include. }\par
						\begin{enumerate}\par
						\item 圆环\par
						\end{enumerate}\par
				\end{enumerate}\par
		\end{enumerate}\par
\end{enumerate}\par

\end{document}